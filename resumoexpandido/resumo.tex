\documentclass[12pt]{article}
\usepackage{times}
\usepackage[utf8]{inputenc}
\usepackage{setspace}
\usepackage{titlesec}
\usepackage[margin=2cm]{geometry} 

\renewcommand{\rmdefault}{ptm}
\renewcommand{\sfdefault}{phv}

\titleformat*{\section}{\fontsize{12}{14}\selectfont\bfseries\center}
\titleformat*{\subsection}{\fontsize{12}{14}\selectfont\bfseries}
\titleformat*{\subsubsection}{\fontsize{12}{14}\selectfont\bfseries}

\begin{document}

\onehalfspacing

\begin{center}
\textbf{TÍTULO DO TRABALHO}
\end{center}

\begin{center}
\textbf{Categoria do Trabalho – Resumo Extendido} \\
\end{center}

\begin{center}
\textit{Nome do Autor\textsuperscript{1}, Nome do Orientador\textsuperscript{2}} \\
\textit{\textsuperscript{1}Instituição do Autor 1, Cidade, País} \\
\textit{\textsuperscript{2}Instituição do Orientador, Cidade, País}
\end{center}

\section*{RESUMO}

Inserir aqui o resumo do trabalho, com fonte Times New Roman, corpo 12, justificado, em parágrafo único, com espaçamento de 1,5 entre as linhas, deve conter no máximo 250 (duzentas e cinquenta) palavras, com breves e concretas informações sobre a justificativa, os objetivos, métodos, resultados e conclusões do trabalho e sem inclusão de tabelas, equações, desenhos e figuras. O arquivo deve ser apresentado em documento de Word, sendo o título do arquivo o mesmo do trabalho. Não deve conter referências bibliográficas. O resumo deve ser apresentado com parágrafo único.

Rascunho: Essa análise tem como objetivo abordar o tema da cibernética e administração e ainda ejdnefiowncweicnerrqjiknm

\textbf{Palavras-chave:} Palavra-chave 1; Palavra-chave 2; Palavra-chave 3; Palavra-chave 4; Palavra-chave 5

\section*{INTRODUÇÃO}

A introdução do trabalho deve ser breve e conter, no máximo, 1000 (um mil) palavras. Com fonte Times New Roman, em corpo 12, justificado, com espaçamento de 1,5 entre as linhas. Justificar o problema estudado de forma clara, utilizando-se revisão de literatura. O último parágrafo deve conter os objetivos do trabalho realizado.

\section*{MÉTODO}

Deve ser concisa, mas suficientemente clara, de modo que o leitor entenda e possa reproduzir os procedimentos utilizados. Deve conter as referências da metodologia de estudo e/ou análises laboratoriais empregadas. Não deve exceder 500 (quinhentas) palavras. Fonte Times New Roman, em corpo 12, justificado, com espaçamento de 1,5 entre as linhas.

\section*{RESULTADOS E DISCUSSÕES}

Inserir os dados obtidos, até o momento, podendo ser apresentados, também, na forma de Tabelas e/ou Figuras. A discussão dos resultados deve estar baseada e comparada com a literatura utilizada no trabalho de pesquisa, indicando sua relevância, vantagens e possíveis limitações. Fonte Times New Roman, em corpo 12, justificado, com espaçamento de 1,5 entre as linhas. Não foi definido um limite máximo de palavras para essa seção, com o objetivo de permitir maior flexibilidade ao(s) autor(es), desde que não seja excedido o limite de seis laudas no total do trabalho.

\section*{CONSIDERAÇÕES FINAIS}

Devem ser elaboradas com o verbo no presente do indicativo, em frases curtas, sem comentários adicionais, e com base nos objetivos e resultados do Resumo Expandido. Não deve exceder 200 (duzentas) palavras, sendo a fonte Times New Roman, em corpo 12, justificado, com espaçamento de 1,5 entre as linhas.

\section*{REFERÊNCIAS}

Devem ser listados apenas os trabalhos mencionados no texto, em ordem alfabética do sobrenome, pelo primeiro autor. Dois ou mais autores, separar por ponto e vírgula. Os títulos dos periódicos não devem ser abreviados. A ordem dos itens em cada referência deve obedecer às normas vigentes da Associação Brasileira de Normas Técnicas – ABNT ou da American Psychological Association (APA).

\end{document}
